\section{Conclusion}\label{sec:conclusion}

In this paper, we defined a prior for multiple quantile regression in which information across quantiles is shared via an adaptive joint shrinkage prior. The structure of the prior is motivated from the penalised non-crossing objective function from \citet{bondell2010noncrossing}, and is shown to imply a quantile state-space representation, named $\QVP$ model, where unknown states are equal to the quantile regression coefficients. This allows for the derivation of efficient sampling methods where the resultant triangular structure of the conditional posterior precision allows for fast computation. We extend the $\QVP$ framework to a non-centred formulation ($\NCQVP$) as well as a post-estimation sparsification algorithm that allow for stronger shrinkage on state variability and sparsity, respectively. With this method, we were able to tackle the issue of quantile crossing through a structured prior that regularises toward the desired parameter sub-space, rather than modifying the likelihood. 
     
A simulation exercise shows that the $\QVP$ priors result in far superior predictive performance and parameter recovery compared to Bayesian methods that estimate quantiles independently. Additionally, crossing is nearly completely eliminated with this approach. For low true quantile variation, the $\NCQVP$ models can also offer large gains over frequentist methods that strictly enforce non-crossing.
    
In the empirical application of a $\QVAR$ on the Euro Area, following \citet{chavleishvili2024forecasting}, we show the practical advantages of the $\QVP$ prior in modelling complex macroeconometric dynamics. We produce quantile forecasts as well as conduct a causal study of the effect of financial shocks to the distribution of industrial production, $\IP$. $\QVP$ models produce very competitive forecasts, often outperforming all models under comparison.
    
For the causal study, we generate impulse response functions ($\QIRF$) of $\IP$ in reaction to shocks to worsening financial conditions. We verify the finding that financial shocks exert markedly asymmetric and persistent effects across the conditional distribution of $\IP$. The $\QVP$ priors produce smoother $\QIRF$s with larger negative effects at the lower tails which are more persistent.

Despite these advantages, there are several avenues by which the method can be improved. First, the paper focuses on implementing the framework to linear quantile regression models. However, the method can be extended to nonlinear settings as well. This extension can enhance the applicability of the $\QVP$ prior framework. Second, we have exclusively focused on the horseshoe prior for modelling the differences across quantiles. The $\QVP$ framework can be used on various other types of shrinkage priors, such as the GIGG prior (see for example \citet{kohns2025flexible}).

%In this paper we extended the findings of \citet{szendrei2023fused} to the Bayesian realm by introducing a novel Bayesian joint quantile regression with quantile specific fused shrinkage. With this method we were able to tackle the issue of quantile crossing through a structure prior, rather than modifying the likelihood. By formulating the problem as a state-space, we are able to use sampling refinements made in the time-varying parameter literature. 

%In our Monte Carlo experiments we showed that the QVP priors consistently outperform independently estimating the quantiles. The QVP prior leads to lower coefficient bias, lower quantile crossing incidence, and tighter confidence bands around the estimated coefficient profiles. The method is particularly potent on small sample sizes, showcasing the methods potential for macroeconometric applications. 

%In the empirical application of QVAR on the Euro Area, following \citet{chavleishvili2024forecasting}, we show the practical advantages of the QVP prior in modelling complex macroeconometric dynamics. We verify the finding that financial shocks exert markedly asymmetric and persistent effects across the conditional distribution of IP. Importantly, we show that the QVP priors produce smoother QIRFs with larger negative effects at the lower tails which are more persistent. In the stress testing exercise we illustrate that the choice of prior greatly influences both the magnitude and duration of a persistent financial stress. 

%Despite these advantages, there are several avenues the method can be improved. First, the paper focuses on implementing the framework to linear quantile regression models. However, the method can be extended to nonlinear settings as well. This extension can enhance the applicability of the QVP prior framework. Second, we have exclusively focused on the Horseshoe prior. The QVP framework can be used on various other types of shrinkage priors, such as the GIGG prior (see for example \citet{kohns2025flexible}).
