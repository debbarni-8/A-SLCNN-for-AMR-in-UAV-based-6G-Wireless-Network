\section{Introduction}
Quantile regression estimates the conditional quantile function of a response variable given a set of covariates. It is a powerful tool for inference on the relationship between response and covariates, especially in the presence of non-linearity in the covariates' impacts across the distribution of the response. While independent estimation of quantiles has been the norm in the applied literature \citep{koenker2005}, the presence of non-monotonically increasing fitted quantile functions, referred to as quantile crossing, remains largely unaddressed. 
%
The probability of observing the crossing problem increases with the number of conditional quantiles estimated and the dimensionality of the covariate set \citet{wang2024composite}. Many constrained optimisation solutions have been suggested in the literature, yet full probabilistic inference remains a challenge, particularly when retaining the assumption of non-parametrically modelling the error distribution. In this paper, we propose a general framework to joint quantile regression where we use the connection between a constrained multiple quantile objective function and an implied negative log posterior to motivate a novel prior structure that penalises crossing of fitted quantiles. We name the suggested prior approach the quantile-varying-parameter ($\QVP$) prior due to the connection to time-varying-parameters ($\mathrm{TVP}$) models popular in the state space literature.  Compared to previous approaches, crossing is penalised via the structure of the prior instead of the structure of the likelihood. In particular, in this paper we:
%
\begin{enumerate}
    \item Establish that the $\QVP$ prior can be regarded as a Bayesian adaptation of the fused quantile lasso model of \citet{jiang2013interquantile} and the composite quantile regression estimator of \citet{zou2008composite} and establish the prior's shrinkage properties
    \item Provide a state-space representation of the $\QVP$ prior, which allows for efficient posterior sampling algorithms 
    \item Compare the the $\QVP$ prior to popular alternatives for Bayesian and frequentist quantile regression approaches in terms of parameter recovery and prediction accuracy in simulated and real-world data experiments
\end{enumerate}
% 
%We establish that the prior can be viewed of a Bayesian generalisation of the fused lasso estimator of \citet{tibshirani2005sparsity} applied to multiple quantile estimation as considered in \citet{jiang2013interquantile} from a frequentist point of view. We show that both the implied likelihood and prior have convenient parametric forms and establish that the joint posterior over the quantile coefficients can be cast into a state-space. This allows for new ways on how to interpret quantile coefficients as well as efficient posterior computation algorithms. It is well known that state-spaces can suffer from slow mixing when variation across states is small. We therefore propose new sampling algorithms that leverage non-centred and mixed centred and non-centred sampling algorithms for improved sampling efficiency. Lastly, we show how one can further improve performance of the proposed quantile models by applying post-processing algorithms to address potential sparsity in the quantile coefficient vector.

%
%A limiting factor in the usefulness of the quantile regression methodology is that, without imposition of monotonicity to a set of fitted conditional quantile functions, one may observe crossing of fitted conditional quantiles. Such extra restrictions necessitate joint-estimation of conditional quantiles. While it is straightforward to extend the quantile regression objective function to encompass multiple quantiles \citet{koenker1978regression}, it is not straightforward in the Bayesian setting due to the lack of a parametric likelihood function. The standard approach for probabilistic inference is to use the asymmetric-Laplace distribution ($\mathcal{ALD}$) as a pseudo likelihood function, and multiple suggestions in the literature extend this approach to the multiple quantile problem via the multivariate asymmetric-Laplace (such as \citet{petrella2019joint,hu2021bayesian,reich2011bayesian} among others). However, the multivariate $\mathcal{ALD}$ provides many computational challenges, and puts the modeller in the awkward position of needing to specify a correlation matrix between quantiles for a likelihood that is assumed to be mis-specified. One may retrieve  non-crossing fitted quantiles via sorting \citep{chernozhukov2010quantile}. However, this does not adjust the coefficients that lead to crossing quantiles. \dk{Some further notes needed on the differences to composite Bayesian quantile regression for multiple quantiles and other approaches, such as the non-parametric approaches, needn't be detailed}.
%


%
The simulated and real data experiments confirm superior inference and prediction performance with our joint prior approach compared to commonly used Bayesian and frequentist models. For the real-world data application, we extend the methods to the Quantile vector-autoregressive (QVAR) model with Euro Area data presented in \citet{chavleishvili2024forecasting}. 
%The QVP framework in a QVAR is particularly important for policy-making, since the coefficients are used to create quantile impulse response functions QIRFs that allow for counter-factual analysis. The results show that financial shocks have an asymmetric effects on industrial production (IP). Importantly we show that while the traditional Bayesian quantile regression produces a sharp dip in the lower quantile of the IP QIRF to a shock in financial stress. In contrast, the proposed methodology, generates impulse responses that are smoother in the quantile dimension, with larger and more persistent impacts at the lower tails. 
%
%The stress-testing setup of \citet{chavleishvili2024forecasting} is also replicated. %The stress-testing is in essence a conditional forecast framework, where the financial stress realisations are given by the researcher. We follow \citet{chavleishvili2024forecasting} and assume that financial stress is high for 6 periods before reverting to median realisations. 
%We show that the choice of prior not only influences the initial impact of shocks but also their duration. 
%
\subsection{Structure of the Paper}
%
In section \ref{subsec:background} we begin by discussing previous approaches to quantile regression as background to this work.
In Section~\ref{sec:motivation} we show that one can view a probabilistic generalisation of the non-crossing constraint as a prior that penalises differences across coefficients of quantiles. \ref{sec:likelihood}, shows that the pseudo-likelihood can be  derived from the objective function of interest. 
We show that this likelihood has a convenient representation as a mixture of normals.
Section~\ref{sec:mcmc} derives an efficient posterior sampling algorithm.
Section~\ref{sec:alt-parameterisation} presents an alternative representation of the $\QVP$ model that offers improved sampling efficiency and shrinkage properties when the data imply low amount of quantile variation.
In Section~\ref{sec:savs}, we discuss post-processing methods for achieving exact sparsity of the $\QVP$ parameter posteriors for improved inference when sparsity in the quantile coefficient vector is suspected.
In Section~\ref{sec:theoretical-properties}, we investigate the theoretical shrinkage properties of the $\QVP$ prior.
We investigate finite sample performance in Section~\ref{sec:simulation}.
We apply the methods presented to multivariate target data, where we   estimate quantile vector autoregressive models ($\QVAR$) with real world data in Section~\ref{sec:application}.
We conclude in Section~\ref{sec:conclusion}.
%
\subsection{Background and Previous Work} \label{subsec:background}
% Paragraph on ALD and shrinkage priors
\citet{yu2001bayesian} established that the commonly used tick-loss quantile regression objective function implies an asymmetric-Laplace distribution ($\mathcal{ALD}$) \citep{kotz2001asymmetric} as a likelihood function. Treated as a working-likelihood,\footnote{Inference on a set of quantile regression coefficients $\beta_q$ for a quantile index $q$, where the percentile $\tau_q \in (0,1)$, is asymptotically equivalent to frequentist treatment of the quantile objective, when treating the $\mathcal{ALD}$ as a working likelihood under conditions discussed in \citet{sriram2013posterior}.} Bayesian inference on quantile regression models has become ubiquitous. The more recent focus being on priors for high dimensional problems \citep{kohns2024horseshoe,alhamzawi2015model,li2010bayesian}. Similar to the literature on normal observation models, those priors are designed to heavily shrink coefficient of noise variables to zero \citep{polson2010shrink}. Yet  these approaches assume that quantile functions are independent, and therefore do not  address the problem of crossing of conditional quantiles.

%
Many probabilistic methods have been put forward to address the issue of crossing fitted quantile functions. These generally fall into the class of semi-parametric \citep{reich2011bayesian,reich2012spatiotemporal,reich2013bayesian,kottas2001bayesian,yang2017joint}, fully non-parametric \citep{scaccia2003bayesian,taddy2010bayesian}, empirical likelihood \citep{lancaster2010bayesian,yang2012bayesian,yang2015quantile} as well as two-step methods \citep{reich2013bayesian,rodrigues2017regression}. 
%

A common perspective taken in the semi-parametric quantile literature is to centre the quantile process for a set of finite number of  percentiles on a fully parametric model which is linked piece-wise via some valid quantile function, such as the normal quantile function \citep{reich2013bayesian}. This approach has some notable drawbacks. One such drawback is that the likelihood may not be available in closed form \citep{reich2011bayesian} necessitating approximation methods. An additional drawback is that conditional posteriors are not available in closed form, thus prohibiting efficient updating via Gibbs MCMC methods \citep{reich2012spatiotemporal,reich2013bayesian}. Computational complexity is also the bottleneck for the empirical likelihood and full non-parametric methods, even in moderate dimensions \citep{rodrigues2017regression}.
\citet{rodrigues2017regression} propose a computationally more convenient approach, in which the first stage of estimating individual quantiles are estimated with an $\mathcal{ALD}$ likelihood. These are combined with a Gaussian process into a valid joint density in a secon step. Their approach maintains valid frequentist coverage. Yet quality of inference heavily depends on the first stage, and joint estimation in the first step has been shown to significantly improve inference even for individual quantiles \citep{bondell2010noncrossing}.

% Closest methods
Closer to our approach are the methods presented in \citet{wu2021bayesian} as well as \citet{wang2024composite} where quantiles are estimated jointly, and information is shared via a prior on the differences. While \citet{wu2021bayesian} take a moment-based approach following \citet{chernozhukov2003mcmc}, \citet{wang2024composite} model the smoothness of neighbouring quantile curves via basis function expansions. In contrast, our proposed $\QVP$ approach connects the logic of difference penalisation of linear quantile models to the non-crossing constrained objective function presented in \citet{bondell2010noncrossing}. Additionally, the $\QVP$ framework allows for efficient posterior computation via Gibbs sampling due to the availability of standard conditional posteriors. While we maintain the assumption that the quantile function is linear in parameters, the method can be extended to use smoothing splines as in \citet{bondell2010noncrossing}. 
%

Estimation frameworks that only allow location shifts of the quantile function are referred to as composite quantile regression ($\CQR$) models \citep{zou2008composite}. Here only the parameter on the intercept identifies differences across quantile functions which leads to non-crossing fitted quantiles. In contrast to the above methods, the $\QVP$ allows for a unifying framework in which the process that models quantile variation is centred on the coefficient vector implied by the $\CQR$ model.    
%Our approach mimics the TVP framework and as such there is a ``common'' component in the coefficient profiles. In the QVP framework, this shared process is how much the covariate shifts every part of the response distribution, i.e. a pure location shift. Estimation frameworks that only allow for pure location shifts is referred to as composite quantile regression \citep{zou2008composite,huang2015bayesian}. 
%Additionally, we center the quantile process on the composite quantile regression model with equal weights \citep{huang2015bayesian} which which 
%We posit that centering on the composite quantile regression allows the QVP framework to inherit much of the desirable theoretical properties.\footnote{\citet{huang2015bayesian} also estimate weights across quantiles, we however, assume the quantiles to all have equal importance.} 

% Frequentist literature
The frequentist literature has proposed many solutions to the quantile crossing problem, where one of the simplest solutions is to sort the fitted quantiles post-estimation \citep{chernozhukov2010quantile}. 
In this paper, interest resides in both prediction and inference on  quantile regression coefficients. We therefore do not consider sorting any further.
%While simple, the procedure does not fix the coefficients that led to crossing quantiles. As such, if the interest is inference on the quantile coefficients one needs a more direct procedure to tackle the crossing problem. 
Most closely related to the QVP framework in the frequentist literature is \citet{szendrei2023fused} who show that, one can re-formulate the exact non-crossing objective function of \citet{bondell2010noncrossing} as a fused lasso model with a particular formulation of the penalisation constant on the fused term.

% Multiple quantile approaches
%\tibi{This paragraph would work better in Emp App where we discuss multivariate quantiles. My thinking is that we don't really push the multivariate quantile lit forward so might be better to have readers more focused on Non-cross}Our approach also differs from the recent literature on multi-variate modelling of multiple quantiles using the multivariate $\mathcal{ALD}$ ($\mathcal{MALD}$). While the $\mathcal{MALD}$ allows modelling of the covariance between quantiles, it does not, without further modification, prohibit crossing of quantiles. And inference on the quantile covariance matrix is complicated due to its non-standard conditional posterior. We therefore do not consider MALD approaches further.


