Crossing of fitted conditional quantiles is a prevalent problem for quantile regression models. We propose a new  Bayesian modelling framework that penalises multiple quantile regression functions toward the desired non-crossing space. We achieve this by estimating multiple quantiles jointly with a prior on variation across quantiles, a fused shrinkage prior with quantile adaptivity. The posterior is derived from a decision-theoretic general Bayes perspective, whose form yields a natural state-space interpretation aligned with Time-Varying Parameter ($\mathrm{TVP}$) models. Taken together our approach leads to a Quantile-Varying Parameter ($\QVP$) model, for which we develop efficient sampling algorithms. 
We demonstrate that our proposed modelling framework provides superior parameter recovery and predictive performance compared to competing Bayesian and frequentist quantile regression estimators in simulated experiments and a real-data application to multivariate quantile estimation in macroeconomics.



%Tibi:
%Crossing of fitted conditional quantiles is a prevalent problem for quantile regression models. We address this by proposing a Bayesian framework that eliminates most of the quantile crossing in one step. We achieve this by formulating a Bayesian framework for estimating multiple quantile jointly and regulating the amount of variation across the quantiles using a fused shrinkage prior with quantile adaptivity. Our approach is grounded in a decision-theoretic general Bayes perspective and has a natural state-space interpretation aligned with Time-Varying Parameter (TVP) models. Taken together our approach leads to a Quantile Varying Parameter (QVP) setup, for which we develop efficient sampling algorithms. To establishing the theoretical finite sample shrinkage properties, we demonstrate that our proposed framework provides superior parameter recovery and predictive performance compared to competing Bayesian and frequentist quantile regression estimators in simulated experiments and a real-data application to multivariate quantile estimation in macroeconomics.

%David:
%Crossing of fitted conditional quantiles is a prevalent problem for quantile regression models. We propose the novel quantile-varying parameter (QVP) framework as a Bayesian solution to penalise multiple quantile regression toward the desired non-crossing space. One can achieve non-crossing conditional quantiles by joint estimation and imposition of fused-shrinkage with quantile adaptivity. This has a natural interpretation of a penalised likelihood over multiple quantiles for which we propose novel Bayesian inference methods. The likelihood is derived from a general Bayes point-of-view which formulates a decision theoretically coherent way to update priors about quantile coefficients. The implied prior structure borrows information across quantiles by adaptively shrinking differences in coefficients over the quantile domain. We show that with appropriate priors, the QVP framework has the advantage of nesting both popular joint-quantile and independent quantile estimation approaches. It has a natural state-space interpretation in line with Time-Varying Parameter (TVP) models for which we propose efficient sampling algorithms. Next to establishing the theoretical finite sample shrinkage properties, we demonstrate that our proposed framework provides superior parameter recovery and predictive performance compared to competing Bayesian and frequentist quantile regression estimators in simulated experiments and a real-data application to multivariate quantile estimation in macroeconomics.


%Crossing of fitted conditional quantiles is a prevalent problem for quantile regression models. One can achieve non-crossing conditional quantiles by joint estimation and imposition of fused-shrinkage with quantile adaptivity. This has a natural interpretation of a penalised likelihood for which we propose novel Bayesian inference methods. The likelihood is derived from a general Bayes point-of-view which formulates a decision theoretically coherent way to update priors about quantile coefficients. We propose a new joint fused shrinkage prior that borrows information across quantiles and allows to adaptively shrink differences in coefficients over the quantile domain to practically eliminate crossing. We show that this prior has a natural state-space interpretation in line with Time-Varying Parameter (TVP) models for which we propose efficient sampling algorithms. Taken together, our approach proposes a framework that penalises variation across quantiles in a state-space setting leading to a Quantile Varying Parameter (QVP) setup. Next to establishing the theoretical finite sample shrinkage properties, we demonstrate that our proposed framework provides superior parameter recovery and predictive performance compared to competing Bayesian and frequentist quantile regression estimators in simulated experiments and a real-data application to multivariate quantile estimation in macroeconomics.  

%The prevalence of crossing quantiles has prompted many to propose methods to estimate quantiles that monotonically increasing. Recently, there has been work to study the impacts such non-crossing constraints have on the estimated coefficients. A key finding of these studies is that one can achieve non-crossing quantiles by jointly estimating the quantiles and imposing fused-shrinkage with quantile specific hyperparameters. This paper builds on this finding and proposes a fully Bayesian framework that shrinks towards non-crossing quantiles. The paper deviates from the standard econometric literature in two ways: 1) we approach quantile estimation via general Bayes approach and focus on minimising a loss function rather than using the asymetric laplace distribution as a working likelihood; and 2) rather than estiamting the quantiles directly we estimate (and shrink) the differences in teh quantiles. We note that the formulateion is very similar to the Time-Varying Parameter formulation frequently used in macroeconometrics. As such for estimation we use samplers that have been shown to be efficient in TVP settings. We show that our proposed framework yields better fits than the estimators frequently used in bayesian and frequentist quantile regression.